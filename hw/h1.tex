\documentclass{article}

\usepackage{algorithm}
\usepackage[noend]{algorithmic}
\usepackage{amsmath, amsthm, amssymb}
\usepackage{enumitem}

\textwidth = 6.5 in
\textheight = 9 in
\oddsidemargin = 0.0 in
\evensidemargin = 0.0 in
\topmargin = 0.0 in
\headheight = 0.0 in
\headsep = 0.0 in
\parskip = 0.2in
\parindent = 0.0in

\begin{document}
\thispagestyle{empty}
\date{}

%\maketitle

%\smallskip


\Large
\begin{center}
	\textbf{Aditya Dhulipala}
	\\*
	\textbf{EE/CSCI 451, Spring 2015 }
	\\*
	\textbf{Homework 1 Submission}
	\\*
\end{center}

\normalsize

\begin{enumerate}[label=\Large\textbf{\arabic*}.]


\item
\begin{enumerate}[label={\arabic*}.]
	\item Superscalar processor
	\item[A.] A processor which has the ability to execute multiple intructions in the same clock cycle is called a superscalar processor
	\\
	\item Row major layout
	\item[A.] TODO answer
	\\
	\item Cache pollution
	\item[A.] TODO answer
	\\
	\item Instruction level parallelism
	\item[A.] 
\end{enumerate}

\item  Prove by induction that the sum of the first $n$ odd positive integers is $n^2$.  Clearly label each step.


\item Suppose you owe someone $v$ cents and must repay using only coins.  You wish to use the fewest coins possible.  If we're using United States currency and have an effectively unlimited supply of each coin type, you can achieve this by repaying with quarters until you owe less than 25 cents, then dimes until you owe less than ten cents, then a nickel if you owe at least five cents, and finally pennies to finish squaring the debt.  A similar strategy also works if you include fifty-cent coins and dollar coins.\\
Suppose instead you were in a country that mints coins with the following values:  one cent, 10 cents, 30 cents, and 40 cents.   Does the strategy of repeatedly giving the most valuable coin whose worth is less than or equal to the debt minimize the number of coins given?  Why or why not?


\item Suppose you have an array $A$ of $n$ integers.  You wish to produce a two-dimensional array $B$ that is $n \times n$, where $B[i][j]$ (for $i < j$) holds the sum of elements $A[i..j]$ (inclusive).  $B[i][j]$, for $i \geq j$, is undefined - you may use those spots in $B$ however you want (or leave garbage/defaults there).
\begin{enumerate}
	\item Write the pseudo-code for an algorithm that takes $A$ as input and produces array $B$ in time $O(n^3)$
	\item Write the pseudo-code for an algorithm that takes $A$ as input and produces array $B$ in time $O(n^2)$.  If you are certain of your answer for this, you may skip part (a) and instead write, for that part, ``see (b).''
	
\end{enumerate}

\end{enumerate}


\end{document}
